
\section{Quality effects on fingerprinted datasets}\label{sec:quality}
To get an insight into how the fingerprinting scheme affects the quality of the data that is disturbed by perturbations, we calculate the mean and variance of the values of an attribute. 
In this section, we first present the analysis and the empirical evaluation of changes in these two measures for each of the fingerprinting schemes described in \Cref{sec:Fingerprinting}. 
Exclusively, for the fingerprinting scheme for categorical data, we use a different quality measurement since the mean and variance do not apply to the categorical values. 
Instead, we count the number of changes in the data introduced by fingerprinting. 

\subsection{AK Scheme}
The procedure of embedding the fingerprint is determined by the owner's secret key $\mathcal{K}$, primary key attribute \textit{P} and controlled by parameters $\gamma$, \textit{v} and $\xi$.
In a dataset with $\eta$ tuples, on average $\eta/\gamma$ tuples are selected for marking. 
In a selected tuple a single bit of a single attribute will be selected for marking. 
Mark value is calculated as a result of applying XOR function on fingerprint bit and pseudorandomly selected mask bit.
The mark bit value will half of the times match the original value, on average, and therefore leave the selected bit unchanged.
Considering that the probability of selecting a value for marking is $1/(\gamma v)$, i.e. one over the number of selected tuples times number of attributes, and the probability that the mark bit differs from the original bit value is $1/2$, we obtain the probability that the value will be selected and changed

\begin{equation}
    P\{L_i=1\} = 1 - P\{L_i=0\} = \frac{1}{2\gamma v}
\end{equation}

where $L_i$ equals 1 if the value of a certain attribute of tuple \textit{i} is selected and changed (in contrast, it equals 0 if the new value doesn't differ from the original value).

In addition, we assume that the original values of an attribute in the dataset are $x_1,x_2,...,x_{\eta}$ and the values of the attribute after fingerprinting are $x_1+\Delta_1,x_2+\Delta_2,...,x_{\eta}+\Delta_{\eta}$. 
\{$\Delta_1,\Delta_2,...,\Delta_\eta$\} represent the error caused by fingerprinting.
They are independent and identically distributed random variables. 
We further assume the representation of $\Delta_i, 1 \leq i \leq \eta$, 

\begin{equation}
\Delta_i = L_i S_i 2^{U_i}
\end{equation}

where $S_i \in \{-1,1\}$ depending on whether the perturbed value is smaller or greater than the original, both with probability 0.5, and $U_i \in \{0,1,...,\xi-1\}$ is uniformly distributed variable representing position of the marked bit. 
\paragraph{Mean}
The mean value of the original attribute values is

\begin{equation}
\overline{x}=(1/\eta)\sum_{i=1}^{\eta}x_i
\end{equation}

and the mean of the attribute values after embedding the fingerprint is

\begin{equation}
\overline{x}'=\overline{x}+\overline{\Delta}
\end{equation} 

where

\begin{equation}
\overline{\Delta}=(1/\eta)\sum_{i=1}^{\eta} \Delta_i
\end{equation}

The expected error of a single attribute value is

\begin{equation}
E[\Delta_i]=\frac{1}{2} L_i 2^{U_i} - \frac{1}{2} L_i 2^{U_i} = 0, \forall i:1 \leq i \leq \eta
\end{equation}

thus the expected error in attribute mean value after embedding the fingerprint is 

\begin{equation}
E[\overline{\Delta}]=0
\end{equation}

\paragraph{Variance}
The variance of the original attribute values is given by 

\begin{equation}
V_x=\frac{1}{\eta}\sum_{i=1}^{\eta} (x_i - \overline{x})^2
\end{equation}

and the variance of the perturbed attribute values after the fingerprint insertion is 

\begin{equation}
V_{x+\Delta}=\frac{1}{\eta}\sum_{i=1}^{\eta}[(x_i+\Delta_i)-(\overline{x}+\overline{\Delta})]^2
\end{equation}

After applying some algebra, the error in variance is given by

\begin{equation}
V_{x+\Delta}-V_x = \frac{1}{\eta} \sum_{i=1}^{\eta}(\Delta_i-\overline{\Delta})^2 + 2\cdot\frac{1}{\eta}\sum_{i=1}^{\eta}(x_i-\overline{x})(\Delta_i-\overline{\Delta})
\end{equation}

The expected error in computing the variance is given by

\begin{equation}
E[V_\Delta]\approx\frac{2^{2\xi}}{6\gamma v \xi}
\end{equation}


\begin{table}[ht]
\centering
\caption{Change in variance introduced by fingerprinting with AK Scheme}
\label{table:AK-mean-var}
\resizebox{\textwidth}{!}{
\begin{tabular}{|lrr|rr|rr|rr|rr|} 
 \hline
 & & $\gamma$ & \multicolumn{2}{c|}{100} & \multicolumn{2}{c|}{50} & \multicolumn{2}{c|}{25} & \multicolumn{2}{c|}{12} \\ 
 & & $\xi$ & \multicolumn{1}{c}{4}&\multicolumn{1}{c|}{8}&\multicolumn{1}{c}{4}&\multicolumn{1}{c|}{8}&\multicolumn{1}{c}{4}&\multicolumn{1}{c|}{8}&\multicolumn{1}{c}{4}&\multicolumn{1}{c|}{8}\\
 \hline
 & \multicolumn{2}{c|}{Expected error in variance} & 0 & 1.4 & 0.01 & 2.7 & 0.02 & 5.5 & 0.04 & 11.4 \\
 \hline
 \textbf{Attribute} & \textbf{Mean} & \textbf{Variance} & & & & & & & & \\
 \hline
 Elevation & 2,959 & 78,391 & 0 &+1& 0 &+1&+1&+5&+1& +9 \\
 \hline
 Aspect & 156 & 12,525 &0 &+1&0 &+1&+1&+5& 0 & +8 \\
 \hline
 Slope & 14 & 56 & 0 &+1 &0 & +3&0 & +5&0  & +11 \\
 \hline
 HD-Hydrology & 269 & 45,177 &0 & +1& 0 &+1 &0 &+2 &+1  &+2  \\
 \hline
 VD-Hydrology & 46 & 3,398 & 0 & +1 & 0 & +2 &0 &+4 &0  &+9  \\
 \hline
 HD-Roadways & 2,350 & 2,431,276 & 0 & +10 & 0 & +10 & \textcolor{red}{-1} & +5 & +2  & +37  \\
 \hline
 Hillshade-9am & 212 & 717 & 0&+1 &0 &+2 &0 &+4 &  0 &+9  \\
 \hline
 Hillshade-noon & 223 & 391 & 0& +1&0 &+2 &0 &+4 &0  & +10 \\
 \hline
 Hillshade-3pm & 143 & 1,465 & 0& +1& 0&+2 &0 &+4 &0  &+8  \\
 \hline
 HD-Fire-Points & 1,980 & 1,753,493 & 0 & \textcolor{red}{-2} & 0 & +5 & 0 & +8 & +1 & +30  \\
 \hline
\end{tabular}
}
\end{table}

\paragraph{Experiments}
Experimental results are obtained by embedding a fingerprint into the Forest dataset. 
We choose the set of following values for parameter $\gamma$ = \{12,25,50,100\}, and $\xi$ = \{4,8\}. 
Table \ref{table:AK-mean-var} contains recorded changes in the variance introduced by fingerprinting for each of the attributes and parameter setting.
The results support the analysis previously made on errors in the mean and variance of the attribute values.
The error in the mean in all of the cases of this experiment was $<0.01$, so only the error in the variance is presented.
The largest changes are expectedly occurring when $\gamma$ is small and $\xi$ is big, i.e. when more tuples are selected and more bits of the value are available for marking. 
The errors in variance between cases with the same $\gamma$ value and different $\xi$ differ significantly, implying that imperceptibility of the fingerprint is highly sensitive to the number of LSBs available for marking.
Original values of the variances, in general, do not affect the relative error of perturbed values.  
In rare cases, the introduced errors result in a decrease of the variance, however, the change is in the same range as in cases where the variance increases. 


\subsection{Block Scheme}

Experimental results are obtained by embedding a fingerprint into the Forest dataset. 
We choose the set of following values for parameter $\beta$ = \{30,25,15,10\}, and $\xi$ = \{4,8\}. 
Table \Cref{table:block-mean} contains recorded changes in mean introduced by fingerprinting for each of the attributes and parameter setting, and \Cref{table:block-var} changes in variance.
\begin{table}[h]
\centering
\caption{Change in mean introduced by fingerprinting with the Block Scheme}
\label{table:block-mean}
\begin{tabular}{|l r|c c| c c| c c| c c|} 
 \hline
 & $\beta$ & \multicolumn{2}{c|}{30} & \multicolumn{2}{c|}{25} & \multicolumn{2}{c|}{15} & \multicolumn{2}{c|}{10} \\ 
 & $\xi$ &4&8&4&8&4&8&4&8\\
 \textbf{Attribute} & \textbf{Mean} & & & & & & & & \\
 \hline
 Elevation & 2959 &  &  &  &  &  &  &  & \\
 \hline
 Aspect & 156 &  &  &  &  &  &  &  & \\
 \hline
 Slope & 14 &  &  &  &  &  & +1 &  & \\
 \hline
 HD-Hydrology & 269 &  &  &  &  &  &  &  & \\
 \hline
 VD-Hydrology & 46 &  &  &  & +1 &  & +1 &  & +1\\
 \hline
 HD-Roadways & 2350 &  &  &  &  &  &  &  & \\
 \hline
 Hillshade-9am & 212 &  &  &  &  &  &  &  & \\
 \hline
 Hillshade-noon & 223 &  &  &  &  &  &  &  & \textcolor{red}{-2}\\
 \hline
 Hillshade-3pm & 143 &  &  &  & \textcolor{red}{-1} &  & \textcolor{red}{-1} &  & \textcolor{red}{-1}\\
 \hline
 HD-Fire-Points & 1980 &  &  &  &  &  &  &  & \\
 \hline
\end{tabular}
\end{table}

\begin{table}[ht]
\centering
\caption{Change in variance introduced by fingerprinting with the Block Scheme}
\label{table:block-var}
\begin{tabular}{|l r|r r|r r|r r|r r|} 
 \hline
 & $\beta$ & \multicolumn{2}{c|}{30} & \multicolumn{2}{c|}{25} & \multicolumn{2}{c|}{15} & \multicolumn{2}{c|}{10} \\ 
 & $\xi$ & \multicolumn{1}{c}{4}&\multicolumn{1}{c|}{8}&\multicolumn{1}{c}{4}&\multicolumn{1}{c|}{8}&\multicolumn{1}{c}{4}&\multicolumn{1}{c|}{8}&\multicolumn{1}{c}{4}&\multicolumn{1}{c|}{8}\\
 \textbf{Attribute} & \textbf{Variance} & & & & & & & & \\
 \hline
 Elevation & 78391 & 0 & +13 & +1 & +15 & +1 & +48 & +1 & +178 \\
 \hline
 Aspect & 12525 & 0 & +7 & 0 & +12 & 0 & +35 & 0 & +127 \\
 \hline
 Slope & 56 & 0 & +12 & 0 & +18 & 0 & +48 & 0 & 0 \\
 \hline
 HD-Hydrology & 45177 & 0 & +6 & +1 & +4 & +1 & +13 & +2 & 0 \\
 \hline
 VD-Hydrology & 3398 & 0 & +10 & 0 & +15 & 0 & +38 & 0 & +87 \\
 \hline
 HD-Roadways & 2431276 & 0 & +3 & 0 & +3 & 0 & +44 & \textcolor{red}{-2} & 0 \\
 \hline
 Hillshade-9am & 717 & 0 & +11 & 0 & +15 & 0 & +41 & 0 & +8 \\
 \hline
 Hillshade-noon & 391 & 0 & +11 & 0 & +16 & 0 & +45 & 0 & +200 \\
 \hline
 Hillshade-3pm & 1465 & 0 & 0 & 0 & +13 & 0 & +35 & 0 & +160 \\
 \hline
 HD-Fire-Points & 1753493 & 0 & 0 & 0 & \textcolor{red}{-4} & 0 & +54 & 0 & +68 \\
 \hline
\end{tabular}
\end{table}

The error is primarily controlled by parameter $\beta$. 
We have previously discussed the expected average number of changes which is described with \Cref{eq:block-changes}. 
$\beta$ in denominator suggests that for smaller $\beta$ there will be more errors in the data introduced by the fingerprint. 
We can see that a larger number of errors expectedly introduces larger errors in variance from the \Cref{table:block-var}.
The smallest error of variance appears for $\beta=30$ and increases for smaller $\beta$ values.

A large difference in the error of variance is noticeable between different values of $\xi$ which defines the number of LSBs available for fingerprinting.
We expect larger errors if more LSBs are available for fingerprinting - bits of more significance are being potentially changed, therefore creating bigger distortions (also assumed from \Cref{eq:block-changes}).
The experiment results in \Cref{table:two-level-mean-var} and \Cref{table:block-mean} confirm this claim. 
Setting $\xi=4$ will result in no or very small errors of variance and mean.
Note that $\xi=8$ may change a value in the dataset up to $\pm 2^7=\pm 128$. 
Considering that most of the values in Forest dataset are in the range of $2^7$, $\xi=8$ is a rather big value. 
For example, in the attribute \textit{Slope} where the mean value is 14 (see \Cref{table:block-mean}), this may cause quite a perceptible modification. 
For attributes with a range of larger values, e.g. \textit{Aspect} with mean 156 or \textit{HD-Roadways} with mean 2350, the error of $\leq \pm 128$ causes slightly less perceptible modifications in single values compared to \textit{Slope}, however the overall variance is certainly affected. 
We see in \Cref{table:block-mean} that error in the mean is introduced only for $\xi=8$ and for attributes with the small mean value.
We have set $\xi=8$ for experiments purely to show the clear distinction of the ways that different values of $\xi$ affect the quality of the data. 
In practice, $\xi$ is set according to the values in the dataset and generally to the smaller values. When $\xi$ is set appropriately, in these experiments $\xi=4$, different values of $\beta$ do not affect the errors in mean and variance too much.

\subsection{Two-level Fingerprinting Scheme}

The embedding process is controlled by $\gamma_1$, $\gamma_2$ and $\xi$.
Using \Cref{eq:total-gamma}, a dataset with $\eta$ tuples will have $ \frac{1}{\gamma}=\frac{1}{\gamma_1}+(1-\frac{1}{\gamma_1})*\frac{1}{\gamma_2}$ tuples selected for marking. Due to the pseudo-random nature of bit marking, the selected bit's original value will half of the times match the mark. 
Thus, the probability that the value will be selected and changed is
\begin{equation}
    P\{L_i=1\} = 1- P\{L_i=0\}=\frac{1}{2\gamma} = \frac{1}{2\gamma_1} +(1-\frac{1}{\gamma_1})*\frac{1}{2\gamma_2}
\end{equation}

where $L_i$ equals 1 if the value of a certain attribute of tuple \textit{i} is selected and changed (in contrast, it equals 0 if the new value doesn't differ from the original value).

The authors of the proposed scheme in their work \cite{guo2006fragile} present the quality effects by measuring mean and variance before and after embedding. 
They show the results from experiments run on a portion of Forest Cover data ($\eta=5,000$) and where fingerprinting is applied to a single attribute - \textit{Elevation}. 
Both mean and variance showed minor absolute alteration ($\approx0.01$) and lead to the conclusion that the scheme preserves the utility of the data. 

We show the empirical results on change in mean and variance in a scenario with entire Forest Cover Type data ($\eta=581,012$). 
For our experiments we consider all 10 numerical attributes for marking ($v=10$) and set parameters as follows: $\gamma_1=\gamma_2=\{10, 25, 50, 100\}$, $\eta=\{4, 8\}$ and $L=96$.

\begin{table}[ht]
\centering
\caption{Change in variance introduced by fingerprinting using the Two-level Fingerprinting Scheme}
\label{table:two-level-mean-var}
\resizebox{\textwidth}{!}{
\begin{tabular}{|lrr|rr|rr|rr|rr|} 
 \hline
 & & $\gamma_1=\gamma_2$ & \multicolumn{2}{c|}{100} & \multicolumn{2}{c|}{50} & \multicolumn{2}{c|}{25} & \multicolumn{2}{c|}{10} \\ 
 & & $\xi$ & \multicolumn{1}{c}{4}&\multicolumn{1}{c|}{8}&\multicolumn{1}{c}{4}&\multicolumn{1}{c|}{8}&\multicolumn{1}{c}{4}&\multicolumn{1}{c|}{8}&\multicolumn{1}{c}{4}&\multicolumn{1}{c|}{8}\\
 \textbf{Attribute} & \textbf{Mean} & \textbf{Variance} & & & & & & & & \\
 \hline
 Elevation & 2959 & 78391 &+1  & +3& +1& +4 & +1 & +4& +1 & +14\\
 \hline
 Aspect & 156 & 12525 &0 & +3 &0& +7& 0& +15&0 & +36\\
 \hline
 Slope & 14 & 56 &0 & +1 &0& +2&0& +4& 0 & +10\\
 \hline
 HD-Hydrology & 269 & 45177 &0&+3 &0& +5&+1& +7& +2 & +10\\
 \hline
 VD-Hydrology & 46 & 3398 & 0& +1&0& +1&0& +3& 0 & +6\\
 \hline
 HD-Roadways & 2350 & 2431276 & \textcolor{red}{-1}& +12&0& +27&0& +31& +1 & +61 \\
 \hline
 Hillshade-9am & 212 & 717 & 0 & 0 & 0& +1&0& +2&0 &+6\\
 \hline
 Hillshade-noon & 223 & 391 & 0& +4& 0& +9&0& +18& 0 & +43\\
 \hline
 Hillshade-3pm & 143 & 1465 &0 &+1 & 0& +3&0& +5&0 & +13\\
 \hline
 HD-Fire-Points & 1980 & 1753493 &+1 & 0 & +1& +12&0& +19& +1 & +34\\
 \hline
\end{tabular}}
\end{table}

The total mark ratio in the embedding process is around $1/\gamma_1 + 1/\gamma_2$. 
\Cref{table:two-level-mean-var} shows that the change in variance increases for larger number of marks, i.e. smaller $\gamma_1$ and $\gamma_2$. 
The change, however, depends on the number of LSBs available for marking $\xi$ as well. 
By choosing a smaller value, the change of variance can be almost completely avoided. The smaller $\xi$ value, however, decreases the robustness of the scheme which needs to be taken into account when choosing the value of this parameter. This is discussed in \Cref{sec:Robustness}.
The mean value of any attribute did not change more than $0.01$ in any of the experiments.
These changes are minuscule enough to hold the claim that the alteration for fingerprinting does not affect the usability of the data.


\subsection{Fingerprinting scheme for categorical data}
The fingerprinting scheme that deals with categorical data requires a different type of measure for data utility since mean and variance are not applicable in this case. One possible measure is the number of changes introduced by marking the data.

We fingerprint the categorical values of the Adult dataset using the fingerprinting scheme for categorical data. Furthermore, we fingerprint the numerical attributes using the AK Scheme. 
\Cref{table:categorical-ak-mean} shows the utility effects on the Adult dataset (which contains 30,162 tuples) introduced by the fingerprinting scheme for categorical data. 
The utility of numerical attributes is still measured by mean and variance, where the difference in the mean is negligible (it does not exceed 0.02 and is therefore excluded from the table). The change in variance introduced by errors for numerical attributes is also rather small, as it was the case with previously presented schemes. 
For each categorical attribute, we count how many changes in values are introduced by the fingerprint.
The Number of values that change in a single categorical attribute is approximately $30,162/(2\gamma v )$. 
For the presented set of parameters, the introduced total number of changes is $<4\%$ of the total number of tuples in the dataset. 
Due to the random nature of fingerprint insertion process, the distributions of attributes are not significantly affected.


\begin{table}[ht]
\centering
\caption{Change in variance and value-flips introduced by fingerprinting with the fingerprinting scheme for categorical data and the AK Scheme, on the Adult dataset}
\label{table:categorical-ak-mean}
\begin{tabular}{|l r|r r|r r|r r|r r|} 
 \hline
 & $\gamma$ & \multicolumn{2}{c|}{50} & \multicolumn{2}{c|}{25} & \multicolumn{2}{c|}{12} & \multicolumn{2}{c|}{6}
 \\ 
 & $\xi$ &2&4&2&4&2&4&2&4 \\

 \textbf{Attribute} & \textbf{Variance} & & & & & & & & \\
 \hline
 Age & 173 & 0 & 0 & 0 & 0 & 0 & 0 & 0 & +0.05 \\
 \hline
 Capital Gain & 54,853,968 & \textcolor{red}{-1} & \textcolor{red}{-3} & \textcolor{red}{-5} & \textcolor{red}{-11} & \textcolor{red}{-23} & \textcolor{red}{-56} & \textcolor{red}{-31} & \textcolor{red}{-67}\\
 \hline
 Capital Loss & 163,457 & 0 & \textcolor{red}{-1} & 0 & \textcolor{red}{-1} & \textcolor{red}{-1} & \textcolor{red}{-2} & \textcolor{red}{-2} & \textcolor{red}{-5} \\
 \hline
 Hours per Week & 144 & 0 & 0 & 0 & 0 & 0 & +0.2 & 0 & +0.3 \\
 \hline
  & & \multicolumn{8}{c|}{\textbf{Value Changes} }\\
\hline
 Workclass & & 26 & 19 & 45 & 45 & 81 & 90 & 165 & 165\\
 \hline
 Education && 26 & 18 & 49 &43 & 83 & 84 & 172 & 173\\
 \hline
 Marital Status && 24 & 24 & 46 & 44 & 101 & 87 & 207 & 189\\
 \hline
 Occupation && 23 & 20 & 44 & 47 & 75 & 73 & 148 & 135\\
 \hline
 Relationship && 22 & 22 & 29 & 41 & 81 & 89 & 175 & 189\\
 \hline
 Race && 19 & 20 & 47 & 51 & 87 & 91 & 160 & 174\\
 \hline
 Sex && 12 & 5 & 19 & 13 & 39 & 25 & 77 & 46\\
 \hline
 Native country && 19 & 21 & 45 & 30 & 94 & 78 & 173 & 164\\
 \hline
 \end{tabular}
 \end{table}
